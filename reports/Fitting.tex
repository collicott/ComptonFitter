\documentclass[]{article}

\usepackage{listings}
\usepackage{color}
\usepackage{amsmath}
\usepackage{csquotes}
\usepackage{subfig}
\usepackage{placeins}
\usepackage{graphicx}
\usepackage{amssymb}
\usepackage{wasysym}
\usepackage{multicol}
\usepackage{subfig}
%\usepackage[caption=false]{subfig}
\usepackage{tabularx}


\definecolor{dkgreen}{rgb}{0,0.6,0}
\definecolor{gray}{rgb}{0.5,0.5,0.5}
\definecolor{mauve}{rgb}{0.58,0,0.82}

	\addtolength{\oddsidemargin}{-.25in}
	\addtolength{\evensidemargin}{-.25in}
	\addtolength{\textwidth}{0.5in}
	
%	\addtolength{\topmargin}{-.875in}
%	\addtolength{\textheight}{1.75in}

\lstset{frame=tb,
	language=Java,
	aboveskip=3mm,
	belowskip=3mm,
	showstringspaces=false,
	columns=flexible,
	basicstyle={\small\ttfamily},
	numbers=none,
	numberstyle=\tiny\color{gray},
	keywordstyle=\color{blue},
	commentstyle=\color{dkgreen},
	stringstyle=\color{mauve},
	breaklines=true,
	breakatwhitespace=true,
	tabsize=3
}

%opening
\title{Proton Polarisability Fitting: CS-APLCON\texttt{++} }
\author{Cristina Collicott}

\begin{document}
	
\maketitle

\begin{abstract}
	A robust fitting routine, known as CS-APLCON\texttt{++}, was designed to extract the proton polarisabilities (both spin and scalar) from Compton scattering data. This fitter is based upon APLCON - a constrained least square fitter by Volker Blobel. As APLCON is written in Fortran, a c\texttt{++} wrapper was written by Andreas Neiser, known as APLCON\texttt{++}. B$\chi$PT, a covariant baryon chiral perturbation calculation from Vladimir Pascalutsa, was used as a theoretical framework. A discussion of different fit results will be presented here. \vspace{8mm}
\end{abstract}

\section{Fitting algorithm: CS-APLCON\texttt{++}}

A new fitting algorithm, known as CS-APLCON\texttt{++}, was designed based upon the previous work by Rory/Phil/Ali. A primary goal was to design a more robust (automated) program to extract the proton polarisabilities from Compton scattering (CS) data. Previous extractions of the spin polarisabilities (SPs) as been centered upon the Pasquini dispersion relation framework. However, the time required to produce data points from this code is roughly 30-60 seconds per data point. Comparatively, the Pascalutsa code is much faster (well below 1 second per point). For this reason, the Pascalutsa code was adopted as a first test.

\subsection{Working with APLCON\texttt{++}}

APLCON\texttt{++} is a constrained least square fitter by Volker Blobel written in Fortran. A c\texttt{++} wrapper was written by Andreas Neiser, known as APLCON\texttt{++}. A new instance of APLCON\texttt{++} can be called in the following manner: \\

\begin{lstlisting}
// Set up APLCON
APLCON::Fit_Settings_t settings = APLCON::Fit_Settings_t::Default;
settings.MaxIterations = 1000;
APLCON aplcon("Polarisabilities", settings);
\end{lstlisting}

\newpage
\noindent APLCON works primarily with measured and unmeasured variables. 

\begin{lstlisting}
// Set initial values
    params fitparam(12.0, 1.9, -4.3, 2.9, -0.02, 2.2);
    constraints ab_sum(13.8, 0.4);
    constraints ab_diff(7.6, 0.9);
    constraints g_0(-1.01, 0.18);
    constraints g_pi(8.0, 1.8);

// Add fit parameters as unmeasured variables
    aplcon.AddUnmeasuredVariable("alpha",fitparam.alpha);
    aplcon.AddUnmeasuredVariable("beta", fitparam.beta);
    aplcon.AddUnmeasuredVariable("E1E1",fitparam.E1E1);
    aplcon.AddUnmeasuredVariable("M1M1",fitparam.M1M1);
    aplcon.AddUnmeasuredVariable("E1M2",fitparam.E1M2);
    aplcon.AddUnmeasuredVariable("M1E2",fitparam.M1E2);
    
// Add measured constraints
    aplcon.AddMeasuredVariable("ab_sum",ab_sum.value, ab_sum.error);
    aplcon.AddMeasuredVariable("ab_diff",ab_diff.value, ab_diff.error);
    aplcon.AddMeasuredVariable("g_0",g_0.value, g_0.error);
    aplcon.AddMeasuredVariable("g_pi",g_pi.value, g_pi.error);
\end{lstlisting}

\noindent \\Finally, constraint functions can be written which range from relatively simple, to very complex. An example constraint, for the Baldin sum rule, could be written in the following manner: \\

\begin{lstlisting}
// Set up constraint lamda functions
    auto sum_constraint  = [] (double a, double b, double sum)  
    {
	    // Baldin sum rule says "a + b = sum"
	    // return 0 = sum - a - b to be minimized
	    return sum - a - b;  
    };

// Apply the constraints
    aplcon.AddConstraint("Baldin", {"alpha", "beta", "alpha_beta_sum"},
								    sum_constraint);
\end{lstlisting}

\noindent \\ Once all variables have been included, and all constraints have been specified, APLCON will try to minimize each constraint: \\

\begin{lstlisting}
// Call fit routine
    const APLCON::Result_t& ra = aplcon.DoFit();
\end{lstlisting}

\subsection{Applied Constraints}

The following section will outline the constraints applied during the fitting routine. These constraints fall under two main categories,

\begin{enumerate}
	\item Data points -- each data point is used as a constraint.
	\item Measured polarisabilities -- the sum and difference of the scaler polarisabilities ($\alpha+\beta$ and $\alpha-\beta$) , and the forward and backward spin polarisabilities ($\gamma_0$ and $\gamma_{\pi}$), are constrained by the experimental values.
\end{enumerate}

\vspace{3mm}

\noindent \textbf{Data points} \\

Data sets can be added to the fitting routine with a text file. Each data set should sit in a file, and a list of files is passed to the fitter. Each new data point is added to the file with the following syntax:

\begin{center}
	90 \quad 288 \quad -0.285 \quad 0.131 \quad Sigma$\_$2x \quad lab\\
	\hspace{1mm}65 \quad 310 \quad -0.212 \quad 0.041 \quad Sigma$\_$3 \quad \hspace{2mm}CM\\
\end{center}

\noindent where the theta, energy, observable, error, observable type, and frame are specified. The fitter will automatically convert between the lab and CM frame, and it is currently capable of using $\Sigma_{2x}$, $\Sigma_{2z}$, $\Sigma_{3}$, and differential cross sections. Each data point is added as a constraint using a lambda function with the following syntax:\\

\begin{lstlisting}
        // setup a lambda function which returns 0
        auto equality_constraint = [] (double experiment, double alpha, 
        double beta, double E1E1, double M1M1, double E1M2, double M1E2) 
        {
	        // call to Pascalutsa code with current fit parameters
	        auto theory = myfit.Fit(datapoint[i].theta, 
										        datapoint[i].energy,
										        alpha, beta, E1E1, M1M1, E1M2, M1E2);       
        
	        // return difference between theory and experiment to minimizer
	        if (datapoint[i].data_type == Sigma_3)
				return experiment - theory.GetSigma3();
				
	        else if (datapoint[i].data_type == Sigma_2x)
				return experiment - theory.GetSigma2x();
				
	        else if (datapoint[i].data_type == Sigma_2z)
				return experiment - theory.GetSigma2z();
				
	        else if (datapoint[i].data_type == Cross)
				return experiment - theory.GetCross();
        };
\end{lstlisting}

\noindent \textbf{Measured polarisabilities} \\

The sum and difference of the scaler polarisabilities ($\alpha+\beta$ and $\alpha-\beta$) are added as a constraint using lambda functions with the following syntax: 

\begin{lstlisting}
// Set up constraint lamda functions
    auto sum_constraint  = [] (double a, double b, double sum)
										    { return sum - a - b; };
    auto diff_constraint = [] (double a, double b, double diff)
										    { return diff - a + b; };
    
// Apply the constraints
    aplcon.AddConstraint("Baldin", {"alpha", "beta", "ab_sum"},  
									sum_constraint);
    aplcon.AddConstraint("Diff",   {"alpha", "beta", "ab_diff"}, 
								    diff_constraint);
\end{lstlisting}

\noindent \\Similarily, the forward and backward spin polarisabilities ($\gamma_0$ and $\gamma_{\pi}$) are added as a constraint using lambda functions with the following syntax: 

\begin{lstlisting}
// Set up constraint lamda functions
    auto g0_constraint   = [] 
	    (double E1E1, double M1M1, double E1M2, double M1E2, double g0 )
		    { return g0 + E1E1 + M1M1 + E1M2 + M1E2;};
		    
    auto gpi_constraint  = [] 
	    (double E1E1, double M1M1, double E1M2, double M1E2, double gpi )
		    { return gpi + E1E1 - M1M1 + E1M2 - M1E2; };
    
// Apply the constraints
    aplcon.AddConstraint("gamma0", 
								    {"E1E1", "M1M1", "E1M2", "M1E2", "gamma_0"},  
								    g0_constraint);
    aplcon.AddConstraint("gammapi",
								    {"E1E1", "M1M1", "E1M2", "M1E2", "gamma_pi"}, 
								    gpi_constraint);

\end{lstlisting}

%\noindent \\The constraints used for this fitting were
%\begin{eqnarray}
%\alpha + \beta &=& \hspace{4mm}13.8 \pm 0.4 \\
%\alpha - \beta &=& \hspace{5.5mm} 7.6 \pm 0.9 \\
%\gamma_0 &=&  \hspace{1mm}-1.01 \pm 0.18 \\
%\gamma_\pi &=& \hspace{5.5mm}8.0 \pm 1.8
%\end{eqnarray}

\section{Data sets}
Three main data sets will be used, 
\begin{enumerate}
	\item $\Sigma_{2x}$ from Martel
	\item $\Sigma_{3}$  from Collicott
	\item $\Sigma_{3}$  from LEGS
\end{enumerate}

\section{Theory predictions}

\begin{table}[h!]
	\centering % used for centering table
	\begin{tabular}{|c|c|c|c|c|c|c|c|c|} % centered columns (4 columns)
		\hline %inserts double horizontal lines
		& \textbf{HDPV} & DPV &$\mathcal{O}(p^{4})_a$ & $\mathcal{O}(p^{4})_b$  & $\mathcal{O}(\epsilon^{3})$ & HB$\chi$PT & \textbf{B$\chi$PT} \\ [0.5ex] % inserts table heading
		\hline\hline % inserts double horizontal line
		$\bar{\gamma}_{E1E1}$ & \textbf{-4.3}  & -3.8 & -5.4 & 1.3 & -1.9 & -1.1 $\pm$ 1.8 & \textbf{-3.3} \\ 
		$\bar{\gamma}_{M1M1}$ & \textbf{2.9}   & 2.9  & 1.4  & 3.3 & 0.4* & 2.2 $\pm$ 1.2 &\textbf{3.0} \\
		$\bar{\gamma}_{E1M2}$ & \textbf{-0.02} & 0.5  & 1.0  & 0.2 & 0.7 & -0.4 $\pm$ 0.4 & \textbf{0.2} \\
		$\bar{\gamma}_{M1E2}$ &\textbf{ 2.2 }  & 1.6  & 1.0  & 1.8 & 1.9 & 1.9 $\pm$ 0.4 & \textbf{1.1} \\
		\hline
		$\gamma_{0}$ 		  & \textbf{-0.8} &  -1.1 & 1.9  & -3.9 & -1.1  & -2.6 & \textbf{-1.0} \\
		$\gamma_{\pi}$ 		  &\textbf{9.4}  &   7.8 & 6.8  & 6.1  & 3.5  & 5.6 & \textbf{7.2} \\
		\hline %inserts single line
	\end{tabular}
\end{table}

\noindent Theoretical predictions of the proton spin polarisabilties are shown for various theoretical frameworks. The first and last column, HDPV and B$\chi$PT, are the Pasquini and Pascalutsa nominal predictions respectively. All polarisabilities are given in units of 10$^{-4}$ fm$^{4}$ with the $\gamma_{\pi}$ $\pi^{0}$ pole term removed.

\section{Previous fits (Can anybody fill in the ?'s)}

\begin{table}[h!]
	\centering % used for centering table
	\begin{tabular}{|c|c|c|c|} % centered columns (4 columns)
		\hline %inserts double horizontal lines
		& LEGS + Martel & Collicott + Martel & LEGS + Martel\\
		& [$10^{-4}$ fm$^{4}$] & [$10^{-4}$ fm$^{4}$] & [$10^{-4}$ fm$^{4}$]\\
		\hline\hline
		$\bar{\gamma}_{E1E1}$ & -3.5 $\pm$ 1.2 				& -5.0 $\pm$ 1.5 \hspace{1mm} & -2.6 $\pm$ 0.8 \\
		$\bar{\gamma}_{M1M1}$ & \hspace{1mm}3.16 $\pm$ 0.85 	& 3.13 $\pm$ 0.88 & 2.7 $\pm$ 0.5 \\
		$\bar{\gamma}_{E1M2}$ & -0.7 $\pm$ 1.2 				& 1.7 $\pm$ 1.7 & ?\\
		$\bar{\gamma}_{M1E2}$ & \hspace{1mm}1.99 $\pm$ 0.29 	& 1.26 $\pm$ 0.43 & ?\\
		\hline
		$\gamma_{0}$ 	& -1.03 $\pm$ 0.18 				& -1.00 $\pm$ 0.18\hspace{1.5mm} & ?\\
		$\gamma_{\pi}$ 	& \hspace{1mm}9.3 $\pm$ 1.6 	& 7.8 $\pm$ 1.8 & ?\\%[0.5ex] 
		$\bar{\alpha} + \bar{\beta}$ & 14.0 $\pm$ 0.4 	& 13.8 $\pm$ 0.4\hspace{1mm} & ?\\
		$\bar{\alpha} - \bar{\beta}$ & \hspace{1mm}7.4 $\pm$ 0.9 	& 6.6 $\pm$ 1.7 & ?\\[0.5ex]
		\hline % inserts double horizontal line
		$\chi^{2}$/dof & 1.05 & 1.25 & ?\\[0.5ex]
		\hline
		Theory & Pasquini & Pasquini & Pascalutsa\\[0.5ex]
		Fit method & (Rory method)? & (Rory method)? & (Rory method)? \\[0.5ex]
		\hline
	\end{tabular}
\end{table}
%

\section{Fitting pseudo data with CS-APLCON\texttt{++}}

A pseudo data generator was written which follows the procedure :

\begin{enumerate}
	\item A random energy/theta value is produced (within user specified limits)
	\item Observable at energy/theta is calculated using theory code
	\item If desired, observable is gauss-smeared to mimic experimental fluctuations.
\end{enumerate}

\newpage
\noindent A snippet of the code for the pseudo generator is given:
\begin{lstlisting}
// 5 Sigma_2x points within E = (280,300) and theta = (60,150) 
pseudo_set.push_back ({5,280.0,300.0,60,150,"Sigma_2x"});

// pol_params  = scaler/spin polarisability values
// true/false  = option to smear observable value (stat. fluctuations)
// percent error = Used to calculate error and sigma for smearing
pseudo.Generate(pseudo_data, pol_params, false, 10.0, theory_code);

\end{lstlisting}

\subsection{Are the two theory codes \enquote{compatible}?}

Pseudo data, generated with known scalar/spin polarisabilities, can be used to study the connection between theory codes. In an ideal world, pseudo data generated with one theory code could be fit using either code, with the resulting polarisabilities being equal (or at least similar). The following section outlines a series of fits using pseudo data generated with the B$\chi$PT theory code (section \ref{Sec:Pseudo1}) and with the HDPV theory code (section \ref{Sec:Pseudo2}). In both cases, the pseudo data is then fit using both codes. In each case, no smearing of the theoretical data points is applied.

\noindent \\Generated data (mimic the MAMI data sets):
\begin{itemize}
	\item 12 $\Sigma_3$ points within E = (270, 305) MeV and $\theta$ = (60, 150)$^{\circ}$
	\item 4 $\Sigma_{2x}$ points within E = (270, 305) MeV and $\theta$ = (60, 150)$^{\circ}$
	\item No smearing, error set to 1$\%$. Just a note: without smearing, the error affects only the convergence tests of the fitter. A discussion of convergence criteria is presented in section \ref{Section:Convergence}.
\end{itemize}

\vspace{5mm}
\subsubsection{Test 1: Pseudo data with B$\chi$PT - no smear/error} \label{Sec:Pseudo1}

\noindent First some nominal numbers:
\begin{table}[h!]
	\centering % used for centering table
	\begin{tabular}{|c|c|c|c|} % centered columns (4 columns)
		\hline %inserts double horizontal lines
		                      & Gen. with B$\chi$PT & Fit with B$\chi$PT & Fit with HDPV\\
		\hline\hline
		$\bar{\alpha}_{E1}$   & 11.2 & 11.18 $\pm$ 0.23 & ? \\
		$\bar{\beta}_{M1}$    & 2.5  & 2.52 $\pm$ 0.19 & ?\\
		$\bar{\gamma}_{E1E1}$ & -3.3 & -3.30 $\pm$ 0.09 & ? \\
		$\bar{\gamma}_{M1M1}$ & 3.0 & 3.00 $\pm$ 0.10 & ? \\
		$\bar{\gamma}_{E1M2}$ & 0.2  & 0.20 $\pm$ 0.15 & ?\\
		$\bar{\gamma}_{M1E2}$ & 1.1 & 1.10 $\pm$ 0.13 & ?\\[0.5ex]
		\hline
	\end{tabular}
\end{table}

\newpage
\noindent Now some I made up ($\gamma_{0}$ and $\gamma_{\pi}$ would be roughly correct):
\begin{table}[h!]
	\centering % used for centering table
	\begin{tabular}{|c|c|c|c|} % centered columns (4 columns)
		\hline %inserts double horizontal lines
		& Gen. with B$\chi$PT & Fit with B$\chi$PT & Fit with HDPV\\
		\hline\hline
		$\bar{\alpha}_{E1}$   & 11.2 & 11.16 $\pm$ 0.22 & ? \\
		$\bar{\beta}_{M1}$    & 2.5  & 2.59 $\pm$ 0.31 & ?\\
		$\bar{\gamma}_{E1E1}$ & -1.0 & -1.00 $\pm$ 0.09 & ? \\
		$\bar{\gamma}_{M1M1}$ & 1.0 & 1.02 $\pm$ 0.16 & ? \\
		$\bar{\gamma}_{E1M2}$ & -2.5  & -2.49 $\pm$ 0.22 & ?\\
		$\bar{\gamma}_{M1E2}$ & 3.5 & 3.48 $\pm$ 0.17 & ?\\[0.5ex]
		\hline
	\end{tabular}
\end{table}

\subsubsection{Test 2: Pseudo data with HDPV - no smear/error} \label{Sec:Pseudo2}

Test 1 is repeated using HDPV as the pseudo data generator. 

\begin{table}[h!]
	\centering % used for centering table
	\begin{tabular}{|c|c|c|c|} % centered columns (4 columns)
		\hline %inserts double horizontal lines
		& Gen. with HDPV & Fit with B$\chi$PT & Fit with HDPV\\
		\hline\hline
		$\bar{\alpha}_{E1}$   & 11.2 & 8.8 $\pm$ 0.14 &  \\
		$\bar{\beta}_{M1}$    & 2.5  & 5.7 $\pm$ 0.25 & ?\\
		$\bar{\gamma}_{E1E1}$ & -3.3 & -2.3 $\pm$ 0.04 & ? \\
		$\bar{\gamma}_{M1M1}$ & 3.0  & 1.3 $\pm$ 0.11 & ? \\
		$\bar{\gamma}_{E1M2}$ & 0.2  & 2.5 $\pm$ 0.11 & ?\\
		$\bar{\gamma}_{M1E2}$ & 1.1  & -0.5 $\pm$ 0.06 & ?\\[0.5ex]
		\hline
	\end{tabular}
\end{table}


\begin{table}[h!]
	\centering % used for centering table
	\begin{tabular}{|c|c|c|c|} % centered columns (4 columns)
		\hline %inserts double horizontal lines
		& Gen. with B$\chi$PT & Fit with B$\chi$PT & Fit with HDPV\\
		\hline\hline
		$\bar{\alpha}_{E1}$   & 11.2 & 10.76 $\pm$ 0.23 & ? \\
		$\bar{\beta}_{M1}$    & 2.5  & 5.32 $\pm$ 0.45 & ?\\
		$\bar{\gamma}_{E1E1}$ & -1.0 & -0.79 $\pm$ 0.06 & ? \\
		$\bar{\gamma}_{M1M1}$ & 1.0  & -1.36 $\pm$ 0.17 & ? \\
		$\bar{\gamma}_{E1M2}$ & -2.5 & 1.9   $\pm$ 0.15 & ?\\
		$\bar{\gamma}_{M1E2}$ & 3.5  & 1.02 $\pm$ 0.08 & ?\\[0.5ex]
		\hline
	\end{tabular}
\end{table}

\subsubsection{Conclusions}

\newpage
\subsection{Are the two theory codes even \enquote{correlated}?}
As shown in Sections \ref{Sec:Pseudo1} and \ref{Sec:Pseudo2}, pseudo data generated with one theory code is not properly reconstructed if the fit is performed with a different code. Although not surprising, it begs the question, \textit{are the theory codes even correlated}? 

\noindent \\Pseudo data was produced using HDPV and B$\chi$PT:
\begin{itemize}
	\item 6 $\Sigma_{3}$ points at (E,$\theta$) = (275 MeV, 70$^\circ$/80$^\circ$/90$^\circ$) and (295 MeV, 70$^\circ$/80$^\circ$/90$^\circ$)
	\item 3 $\Sigma_{2x}$ points at (E,$\theta$) = (295 MeV, 70$^\circ$/80$^\circ$/90$^\circ$)
	\item No smearing, error set to 1$\%$. 
	\item $\bar{\alpha}_{E1}$ and $\bar{\beta}_{M1}$ fixed to 11.2 and 2.5
	\item $\bar{\gamma}_{M1M1}$, $\bar{\gamma}_{E1M2}$, $\bar{\gamma}_{M1E2}$ fixed to nominal HDPV (2.9, -0.02, 2.2)
	\item \textbf{$\bar{\gamma}_{E1E1}$ is varied between -6.0 and 6.0 with a step size of 1}
\end{itemize}

\noindent \\The pseudo data is then fit using B$\chi$PT. 

\noindent \vspace{5mm}\\Because only $\bar{\gamma}_{E1E1}$ is varied, the traditional $\gamma_{0}$ and $\gamma_{\pi}$ constraints do not apply. For this reason, a small adjustment was made to the fitting routine. Specifically, the difference between the generated $\bar{\gamma}_{E1E1}$ and the nominal value of -4.3 is calculated:
\begin{equation}
\delta_{E1E1} = -4.3 \hspace{1mm} - \bar{\gamma}_{E1E1}\text{ (generated)},
\end{equation}
and the $\gamma_{0}$ and $\gamma_{\pi}$ constraints are modified such that, 
\begin{equation}
\gamma_{0} = -(\bar{\gamma}_{E1E1}+ \delta_{E1E1}) - \bar{\gamma}_{M1M1} - \bar{\gamma}_{E1M2} - \bar{\gamma}_{M1E2}, 
\end{equation}
\begin{equation}
\gamma_{\pi} = -(\bar{\gamma}_{E1E1}+ \delta_{E1E1}) + \bar{\gamma}_{M1M1} - \bar{\gamma}_{E1M2} + \bar{\gamma}_{M1E2}. 
\end{equation}

\noindent \vspace{5mm}\\ Figures (a) through (f) show the reconstructed spin/scalar polarisabilities as $\bar{\gamma}_{E1E1}$ is varied. As only $\bar{\gamma}_{E1E1}$ is varied, this is a relatively simple test of the correlation between HDPV and B$\chi$PT codes. In an ideal world, only the reconstructed $\bar{\gamma}_{E1E1}$ would vary, while the others remain constant. This appears to be true for pseudo data generated with, and fit using, B$\chi$PT. However, this is not the case when data is generated with HDPV and fit using B$\chi$PT. This suggests a strong model dependance between the two theory codes (with interplay between the variables).

\noindent \\ \textbf{Conclusion:} Let's take the words \enquote{model independant} out of our vocabulary.

\newpage
\begin{figure}[h!]
	\centering
	\subfloat[$\bar{\gamma}_{E1E1}$ (reconstructed with B$\chi$PT) is shown as a function of $\bar{\gamma}_{E1E1}$  generated with either HDPV (black) or B$\chi$PT (blue).  $\bar{\gamma}_{M1M1}$, $\bar{\gamma}_{E1M2}$, $\bar{\gamma}_{M1E2}$ are fixed to nominal HDPV: 2.9, -0.02, and 2.2 respectively. The red line shows the expected (y=x) correlation.] {\includegraphics[width=0.85\linewidth]{/home/cristina/ComptonFitter/results/Pseudo_E1E1.pdf}} \\
	
	\subfloat[$\bar{\gamma}_{M1M1}$ (reconstructed with B$\chi$PT) is shown as a function of $\bar{\gamma}_{E1E1}$  generated with either HDPV (black) or B$\chi$PT (blue).  $\bar{\gamma}_{M1M1}$, $\bar{\gamma}_{E1M2}$, $\bar{\gamma}_{M1E2}$ are fixed to nominal HDPV: 2.9, -0.02, and 2.2 respectively. The green line shows the nominal/generated value of $\bar{\gamma}_{M1M1}$.] {\includegraphics[width=0.85\linewidth]{/home/cristina/ComptonFitter/results/Pseudo_M1M1.pdf}}
\end{figure}

\newpage
\begin{figure}[h!]
	\ContinuedFloat
	\centering
	\subfloat[$\bar{\gamma}_{E1M2}$ (reconstructed with B$\chi$PT) is shown as a function of $\bar{\gamma}_{E1E1}$  generated with either HDPV (black) or B$\chi$PT (blue).  $\bar{\gamma}_{M1M1}$, $\bar{\gamma}_{E1M2}$, $\bar{\gamma}_{M1E2}$ are fixed to nominal HDPV: 2.9, -0.02, and 2.2 respectively. The green line shows the nominal/generated value of $\bar{\gamma}_{E1M2}$.] {\includegraphics[width=0.85\linewidth]{/home/cristina/ComptonFitter/results/Pseudo_E1M2.pdf}} \\
	
	\subfloat[$\bar{\gamma}_{M1E2}$ (reconstructed with B$\chi$PT) is shown as a function of $\bar{\gamma}_{E1E1}$  generated with either HDPV (black) or B$\chi$PT (blue).  $\bar{\gamma}_{M1M1}$, $\bar{\gamma}_{E1M2}$, $\bar{\gamma}_{M1E2}$ are fixed to nominal HDPV: 2.9, -0.02, and 2.2 respectively. The green line shows the nominal/generated value of $\bar{\gamma}_{M1E2}$.] {\includegraphics[width=0.85\linewidth]{/home/cristina/ComptonFitter/results/Pseudo_M1E2.pdf}} \\
\end{figure}


\newpage
\begin{figure}[h!]
	\ContinuedFloat
	\centering
	\subfloat[$\bar{\alpha}_{E1}$ (reconstructed with B$\chi$PT) is shown as a function of $\bar{\gamma}_{E1E1}$ (produced with HDPV). $\bar{\gamma}_{M1M1}$, $\bar{\gamma}_{E1M2}$, $\bar{\gamma}_{M1E2}$ are fixed to nominal HDPV: 2.9, -0.02, and 2.2 respectively. ] {\includegraphics[width=0.85\linewidth]{/home/cristina/ComptonFitter/results/Pseudo_alpha.pdf}} \\
	
	\subfloat[$\bar{\beta}_{E1}$ (reconstructed with B$\chi$PT) is shown as a function of $\bar{\gamma}_{E1E1}$ (produced with HDPV). $\bar{\gamma}_{M1M1}$, $\bar{\gamma}_{E1M2}$, $\bar{\gamma}_{M1E2}$ are fixed to nominal HDPV: 2.9, -0.02, and 2.2 respectively. ] {\includegraphics[width=0.85\linewidth]{/home/cristina/ComptonFitter/results/Pseudo_beta.pdf}} \\
\end{figure}

\newpage
\subsection{What about at lower energies?}

\begin{figure}[h!]
	\centering
	\subfloat[$\bar{\alpha}_{E1}$] {\includegraphics[width=0.5\linewidth]{/home/cristina/ComptonFitter/results/Pseudo_alpha_lowE.pdf}}
	\subfloat[$\bar{\beta}_{E1}$] {\includegraphics[width=0.5\linewidth]{/home/cristina/ComptonFitter/results/Pseudo_beta_lowE.pdf}}\\
	\subfloat[$\bar{\alpha}_{E1}$] {\includegraphics[width=0.5\linewidth]{/home/cristina/ComptonFitter/results/Pseudo_E1E1_lowE.pdf}}
	\subfloat[$\bar{\beta}_{E1}$] {\includegraphics[width=0.5\linewidth]{/home/cristina/ComptonFitter/results/Pseudo_M1M1_lowE.pdf}}\\	
	\subfloat[$\bar{\alpha}_{E1}$] {\includegraphics[width=0.5\linewidth]{/home/cristina/ComptonFitter/results/Pseudo_E1M2_lowE.pdf}}
	\subfloat[$\bar{\beta}_{E1}$] {\includegraphics[width=0.5\linewidth]{/home/cristina/ComptonFitter/results/Pseudo_M1E2_lowE.pdf}}\\		
\end{figure}

\newpage
\section{Fitting real data with CS-APLCON\texttt{++}}

The following section will outline numerous fits which were performed with the new fitting routine, CS-APLCON\texttt{++}. In section \ref{Section:PhilTest}, a simple test to replicate the results of Phil's thesis will be shown. In section \ref{Section:LEGSTest}

\subsection{Test 1: E1E1 with Martel $\Sigma_{2x}$}\label{Section:PhilTest}

Within the $\Delta$ region, $\Sigma_{2x}$ shows a strong sensitivity to $\bar{\gamma}_{E1E1}$ (while being essentially insensitive to $\bar{\gamma}_{M1M1}$). Because of this, a simple extraction of $\bar{\gamma}_{E1E1}$ is possible using only $\Sigma_{2x}$ data. This test was done in Phil's thesis (with Pasquini's HDPV theory code) and determined a value of $\bar{\gamma}_{E1E1}$ to be,
%
\begin{equation}
\bar{\gamma}_{E1E1} = -4.3 \pm 1.5 
\end{equation}
%
where $\alpha$ and $\beta$ were the old PDG values (12.1 and 1.6 respectively), and $\bar{\gamma}_{M1M1}$/$\bar{\gamma}_{E1M2}$/$\bar{\gamma}_{M1E2}$ were fixed to the HDPV nominal values (given in Section 3). \\

\noindent \\This test was replicated with CS-APLCON\texttt{++} using:
\begin{itemize}
	\item Pasquini (HDPV) and Pascalutsa (B$\chi$PT) as fitting theory
	\item New and old PDG values for $\bar{\alpha}_{E1}$ and $\bar{\beta}_{M1}$
	\item Fixed values of $\bar{\gamma}_{M1M1}$/$\bar{\gamma}_{E1M2}$/$\bar{\gamma}_{M1E2}$ to HDPV and B$\chi$PT nominal values
\end{itemize}

\begin{table}[h!]
	\centering % used for centering table
	\begin{tabular}{|c|c|c|c|c|} % centered columns (4 columns)
		\hline %inserts double horizontal lines
		$\bar{\gamma}_{E1E1}$ & \textbf{-4.66} $\pm$ \textbf{1.59}  & \textbf{-4.66} $\pm$ \textbf{1.58} & \textbf{-4.80} $\pm$ \textbf{1.56} & \textbf{-4.80} $\pm$ \textbf{1.56}\\ [0.5ex]
		Fitting Theory & HDPV &  HDPV & HDPV & HDPV \\[0.5ex]		
		\hline
		\hline %inserts double horizontal lines
		$\bar{\gamma}_{E1E1}$ & \textbf{-3.68} $\pm$ \textbf{1.29}  & \textbf{-3.65} $\pm$ \textbf{1.31} & \textbf{-3.73} $\pm$ \textbf{1.27} & \textbf{-3.70} $\pm$ \textbf{1.28}\\ [0.5ex]
		Fitting Theory & B$\chi$PT &  B$\chi$PT & B$\chi$PT & B$\chi$PT\\[0.5ex]
		\hline
		\hline		
		$\bar{\alpha}_{E1}$   (fixed) & 12.1 & 11.2 & 12.1 & 11.2 \\
		$\bar{\beta}_{M1}$   (fixed) & 1.6 & 2.5 & 1.6 & 2.5\\
		\hline
		$\bar{\gamma}_{M1M1}$ (fixed) & 2.9   & 2.9  & 3.0  & 3.0 \\
		$\bar{\gamma}_{E1M2}$ (fixed) & -0.02 & -0.02 & 0.2 & 0.2  \\
		$\bar{\gamma}_{M1E2}$ (fixed) & 2.2 & 2.2 & 1.1 & 1.1 \\	[0.5ex]
		\hline
		Fixed to & HDPV and & HDPV and & B$\chi$PT and & B$\chi$PT and \\
		nominal  & old PDG & new PDG & old PDG & new PDG \\
		\hline
	\end{tabular}
\end{table}

\noindent \\ This gives a rough value of the extraction to be,
\begin{equation}
\bar{\gamma}_{E1E1} \approx -4.7 \pm 1.6 \hspace{2mm} \text{(fit)}, \hspace{5mm} \bar{\gamma}_{E1E1} = -4.3 \text{ (nominal) }
\end{equation}
for the HDPV fitting theory and,
\begin{equation}
\bar{\gamma}_{E1E1} \approx -3.7 \pm 1.3 \hspace{2mm} \text{(fit)}, \hspace{5mm} \bar{\gamma}_{E1E1} = -3.3 \text{ (nominal) }
\end{equation}
for fitting with B$\chi$PT. While the spread in values for each fitting theory (HDPV or B$\chi$PT) is small, the separation between the two methods is relatively large (on the order of 1$\times$10$^{-4}$ fm$^{4}$). This same separation exists between the nominal values of $\bar{\gamma}_{E1E1}$. This could suggest a possible scale for the model dependance associated with the fitting method. 

\newpage

\subsection{Test 2: Spin Pols with Collicott $\Sigma_{3}$ and Martel $\Sigma_{2x}$}\label{Section:CollicottTest}

A fit was performed using $\Sigma_{2x}$ results from Martel and $\Sigma_{3}$ results from Collicott. Results from CS-APLCON\texttt{++}, along with previous fits and nominal values from the HDPV and B$\chi$PT theories are shown in the table below.

\begin{table}[h!]
	\centering % used for centering table
	\begin{tabular}{|c|cc|c|c|c|} % centered columns (4 columns)
		\hline
		& Nominal & Nominal & Previous Fit & New fit & New fit\\
		\hline %inserts double horizontal lines
		$\bar{\gamma}_{E1E1}$ & -4.3 & -3.3 & -5.0 $\pm$ 1.5  & -3.98 $\pm$ 0.95 & -5.15 $\pm$ 1.12 \\ 
		$\bar{\gamma}_{M1M1}$ & 2.9 & 3.0   & 3.13 $\pm$ 0.88 &  2.76 $\pm$ 0.52 & 3.07 $\pm$ 0.50\\
		$\bar{\gamma}_{E1M2}$ & -0.02 & 0.2 & 1.7 $\pm$ 1.7   &  0.20 $\pm$ 1.17 & 1.76 $\pm$ 1.32\\
		$\bar{\gamma}_{M1E2}$ & 2.2 & 1.1   & 1.26 $\pm$ 0.43 &  2.03 $\pm$ 0.56 & 1.32 $\pm$ 0.52\\[0.5ex]
				$\chi^{2}$    &     &       &  ?              & 19.405 & 17.566 \\		
		\hline
		Fitting Theory & HDPV & B$\chi$PT & HDPV & B$\chi$PT& HDPV\\
		%		Fitting Method &      &           & Rory & CS-APLCON\texttt{++} \\
		\hline
	\end{tabular}
\end{table}

\subsection{Test 3: Spin Pols with LEGS $\Sigma_{3}$ and Martel $\Sigma_{2x}$}\label{Section:LEGSTest}

A fit was performed using $\Sigma_{2x}$ results from Martel and $\Sigma_{3}$ results from LEGS (only data below the double pion threshold were used). Results from CS-APLCON\texttt{++}, along with previous fits and nominal values from the HDPV and B$\chi$PT theories are shown in the table below.

\begin{table}[h!]
	\centering % used for centering table
	\begin{tabular}{|c|cc|c|c|c|} % centered columns (4 columns)
		\hline
		                    & Nominal & Nominal & Previous Fit & New fit & New fit\\
		\hline %inserts double horizontal lines
		$\bar{\gamma}_{E1E1}$ & -4.3 & -3.3 & -3.5 $\pm$ 1.2 & -0.38 $\pm$ 0.78 & 1.09 $\pm$ 0.70 \\ 
		$\bar{\gamma}_{M1M1}$ & 2.9 & 3.0 & 3.16 $\pm$ 0.85 & 2.40 $\pm$ 0.55 & 1.09 $\pm$ 0.70\\
		$\bar{\gamma}_{E1M2}$ & -0.02 & 0.2 & -0.7 $\pm$ 1.2 & -2.45 $\pm$ 0.94 & 1.09 $\pm$ 0.70\\
		$\bar{\gamma}_{M1E2}$ & 2.2 & 1.1 & 1.99 $\pm$ 0.29 & 1.45 $\pm$ 0.53 & 1.09 $\pm$ 0.70\\[0.5ex]
		\hline
		Fitting Theory & HDPV & B$\chi$PT & HDPV & B$\chi$PT& HDPV\\
%		Fitting Method &      &           & Rory & CS-APLCON\texttt{++} \\
		\hline
	\end{tabular}
\end{table}




















\newpage 
\subsection{A note on convergence tests of CS-APLCON\texttt{++}}\label{Section:Convergence}

Fitting within APLCON\texttt{++} is an iterative process. Each iteration is tested against \enquote{convergence criteria}. The default convergence criteria of APLCON\texttt{++} are very strict, requiring a $\partial\chi^{2}$ of 10$^{-5}$. Although it is possible to achieve this level of $\partial\chi^{2}$ for some applications, this is very difficult considering we are attempting to solve a non-linear multi-dimensional problem. To be sure that the convergence criteria can be reduced, two fits were done with different convergence tests. \\

\noindent Fits were performed with the Pascalutsa theory code, using LEGS, Collicott, and Martel data (below 2$\pi$ threshold). Each fit had a different $\partial\chi^{2}$ for the convergence criteria. The results appear below for LEGS and Martel data:
 
\begin{table}[h!]
	\centering % used for centering table
	\begin{tabular}{|c|c|c|c|} % centered columns (4 columns)
		\hline
		Convergence Criteria, $\partial\chi^{2}$ & 1.0 $\times$ 10$^{-4}$ & 0.5$\times$10$^{-3}$ & 1.0$\times$10$^{-2}$\\
		\hline %inserts double horizontal lines
		$\bar{\gamma}_{E1E1}$ & -0.37 $\pm$ 0.78 & -0.40 $\pm$ 0.77 & -0.38 $\pm$ 0.78 \\ 
		$\bar{\gamma}_{M1M1}$ & 2.39 $\pm$ 0.55 & 2.38 $\pm$ 0.55 & 2.40 $\pm$ 0.55 \\
		$\bar{\gamma}_{E1M2}$ & -2.45 $\pm$ 0.94 & -2.41 $\pm$ 0.94 & -2.45 $\pm$ 0.94 \\
		$\bar{\gamma}_{M1E2}$ & 1.44 $\pm$ 0.54 & 1.46 $\pm$ 0.54 & 1.45 $\pm$ 0.53 \\[0.5ex]
		$\chi^{2}$ & 115.441 & 115.443 & 115.441 \\
		\hline
		Fitting Theory & B$\chi$PT & B$\chi$PT & B$\chi$PT \\
		\hline
	\end{tabular}
\end{table}

\noindent and appear below for Collicott and Martel data:

\begin{table}[h!]
	\centering % used for centering table
	\begin{tabular}{|c|c|c|c|} % centered columns (4 columns)
		\hline
		Convergence Criteria, $\partial\chi^{2}$ & 1.0 $\times$ 10$^{-4}$ & 0.5$\times$10$^{-3}$ & 1.0$\times$10$^{-2}$\\
		\hline %inserts double horizontal lines
		$\bar{\gamma}_{E1E1}$ & -3.97 $\pm$ 0.96 & -3.99 $\pm$ 0.96 & -3.98 $\pm$ 0.95 \\ 
		$\bar{\gamma}_{M1M1}$ & 2.76 $\pm$ 0.52 & 2.76 $\pm$ 0.52 & 2.77 $\pm$ 0.52 \\
		$\bar{\gamma}_{E1M2}$ & 0.21 $\pm$ 1.18 & 0.21 $\pm$ 1.17 & 0.20 $\pm$ 1.17 \\
		$\bar{\gamma}_{M1E2}$ & 2.02 $\pm$ 0.56 & 2.03 $\pm$ 0.56 & 2.03 $\pm$ 0.56 \\[0.5ex]
		$\chi^{2}$ & 19.405 & 19.405 & 19.405 \\
		\hline
		Fitting Theory & B$\chi$PT & B$\chi$PT & B$\chi$PT \\
		\hline
	\end{tabular}
\end{table}

\noindent Using less restrictive convergence criteria results in a very small shift in the final spin polarisabilities (on the order of $\approx$ 0.01). The time to perform a fit, however, increases dramatically for more restrictive criteria. For the sake of this report, a convergence criteria of $\partial\chi^{2}$ of 1$\times$10$^{-2}$ was applied.

\end{document}